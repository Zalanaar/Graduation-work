\documentclass[a4paper,english,russian]{G2-105}
\usepackage[T1]{fontenc}
\usepackage{listings}
\usepackage{graphicx}
\usepackage{longtable}
\usepackage{booktabs}
\usepackage{standalone}
\usepackage{newclude}
\usepackage[final]{pdfpages}
\usepackage{multirow}
\usepackage{sansmath} 
\sansmath 
\usepackage[math]{blindtext}

\VSTUSetDocumentNumbersPrefix{}
\VSTUSetDocumentCode{Номер документа бакалавра}
\VSTUSetDocumentTypeDative{выпускной работе бакалавра}
\VSTUSetDocumentTypeGenitive{выпускную работу бакалавра}
\VSTUSetInitialData{задание, выданное научным руководителем с кафедры САПРиПК, утвержденное приказом ректора}
\VSTUSetPZContents{
\begin{VSTUList}
\ulitem{Введение}
\ulitem{1 Задача прогнозирования}
\ulitem{Цели и задачи исследования}
\ulitem{2 Изучение конкурирующих решений}
\ulitem{Выводы}
\ulitem{3 Проектирование решения}
\ulitem{Выводы}
\ulitem{4 Реализация программы}
\ulitem{Выводы}
\ulitem{Заключение}
\ulitem{Список использованных источников}
\ulitem{Приложение А - Техническое задание}
\end{VSTUList}
}
\begin{document}
\VSTUSetOrder{номер}{дата}{месяц}{год}
\VSTUSetFaculty{Электроники и вычислительной техники}
\VSTUSetDepartment{Системы автоматизированного проектирования и ПК}
\VSTUSetDepartmentCode{код кафедры}
\VSTUSetDirection{Возможно сокращенное}
\VSTUSetHeadOfDepartment{Зав. кафедрой САПРиПК}{д.т.н., проф.}{М.В. Щербаков}{Щербаков Максим Владимирович}
\VSTUSetDirector{Зав. кафедрой САПРиПК}{д.т.н}{М.В.Щербаков}{Щербаков Максим Владимирович}
\VSTUSetFacilityExpert{}{}{}{}
\VSTUSetStandardsAdviser{доцент каф.САПРиПК}{}{О.А.Шабалина}{Шабалина Ольга Аркадьевна}
\VSTUSetStudent{ИВТ-461}{А.Д.Евтеев}{Евтеев Артем Дмитриевич}
\VSTUSetTitle{Разработка алгоритма прогнозирования перемещения человека в городской среде на основе анализа геораспределенных данных}
\VSTUSetTitleEng{Support enumerations for C++ programm language in the question type CorrectWriting}
\VSTUInitializePZ
\abstract{Аннотация}
\par Документ представляет собой пояснительную записку к выпускной работе бакалавра на тему «Разработка алгоритма прогнозирования перемещения человека в городской среде на основе анализа геораспределенных данных», выполненную студентом группы ИВТ-461, Евтеевым Артемом Дмитриевичем.
\par В данной работе были реализованы генерация данных, для дальнейшего прогноза на их основе и алгоритм, прогнозирующий перемещение человека в черте города Волгограда.
\par Объём пояснительной записки составил \totalpages~страниц и включает \totalfigures~рисунков и \totaltables~таблицы. 
\par Ключевые слова: геораспределенные данные, прогноз, машинное обучение, Random Forest.
\tableofcontents
\newpage



\appendixdocument{Техническое задание}
\end{document}